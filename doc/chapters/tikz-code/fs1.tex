\begin{figure}[htbp]
\centering
\begin{tikzpicture}[
    block/.style={draw, minimum width=2.2cm, minimum height=1cm, align=center},
    arrow/.style={->, >=stealth, thick}
]
    \node[block, fill=blue!20] (sb) at (0, 0) {超级块\\Block 0};
    \node[block, fill=orange!20] (itable) at (2.5, 0) {inode表\\Block 1};
    \node[block, fill=green!20] (rootdata) at (5, 0) {根目录数据\\Block 2};
    \node[block, fill=gray!20] (reserved) at (7.5, 0) {保留区\\Block 3-9};
    \node[block, fill=yellow!20] (data) at (10, 0) {数据区\\Block 10+};
    
    \node[below=0.3cm] at (sb.south) {\footnotesize 偏移 0};
    \node[below=0.3cm] at (itable.south) {\footnotesize 偏移 512};
    \node[below=0.3cm] at (rootdata.south) {\footnotesize 偏移 1024};
\end{tikzpicture}
\caption{简易文件系统磁盘布局}
\label{fig:fs-layout}
\end{figure}
