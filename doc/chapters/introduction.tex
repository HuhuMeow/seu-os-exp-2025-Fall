%% introduction.tex 
%% ------------------
%% Author: HuhuMeow (Huhu_Miao)
%% License: MIT
\newcommand{\sysc}{\texttt{kernel/sys.c}}
\newcommand{\syscallh}{\texttt{include/linux/syscall.h}}
\newcommand{\syscalltbl}{\texttt{scripts/syscall.tbl}}
\section{实验环境简述}
\par 本次实验采用\texttt{Arch Linux ARM aarch64},内核版本为{Linux 6.18.0-1-aarch64-ARCH}\footnote{6.18版本的内核在2025年11月30日被确认为LTS版本,会支持到2027年12月。}。虚拟机版本为VMware Fusion 13.6.4。
\par 我注意到,在\texttt{archinstall}的帮助下,Arch Linux 的安装并不比其他发行版更难。由于 Arch Linux 的轻量化,它在虚拟机中的启动速度也令我满意,这对于需要频繁重启的实验场景友好。出于以上原因,我选择了 Arch Linux 。
\section{Linux进程管理及其扩展}
\subsection{实验内容}
\par 实现一个系统调用\texttt{hide},使得可以根据指定的参数隐藏进程,使用户无法使用\texttt{ps}或\texttt{top}观察到进程状态。
\subsection{具体要求}
\begin{enumerate}
	\item 实现系统调用\texttt{int hide(pid\_t pid, int on)},在进程\texttt{pid}有效的前提下,如果\texttt{on}置\texttt{1},进程被隐藏,用户无法通过\texttt{ps}或\texttt{top}观察到进程状态;如果\texttt{on}置\texttt{0}且此前为隐藏状态,则恢复正常状态。
	\item 考虑权限问题,只有\texttt{root}用户才能隐藏进程。 
	\item 设计一个新的系统调用\texttt{int hide\_user\_processes(uid\_t uid, char *binname)},参数\texttt{uid}为用户ID号,当\texttt{binname}参数为\texttt{NULL}时,隐藏该用户的所有进程;否则,隐藏二进制映像名为\texttt{binname}的用户进程。该系统调用应与\texttt{hide}系统调用共存。
  \item 在\texttt{/proc}目录下创建一个文件\texttt{/proc/hidden},该文件可读可写,对应一个全局变量\texttt{hidden\_flag},当\texttt{hidden\_flag}为0时,所有进程都无法隐藏,即便此前进程被\texttt{hide}系统调用要求隐藏。只有当\texttt{hidden\_flag}为1时,此前通过\texttt{hide}调用要求被屏蔽的进程才隐藏起来
	\item 在\texttt{/proc}目录下创建一个文件\texttt{/proc/hidden\_process},该文件的内容包含所有被隐藏进程的\texttt{pid},各\texttt{pid}之间用空格分开
\end{enumerate}
\newpage
\subsection{实现系统调用\texttt{hide}}
\par 首先要明确,添加一个系统调用可以分为三个部分:函数实现、添加函数声明、注册系统调用号。其中,添加函数实现需要修改\texttt{kernel/sys.c},添加函数声明需要修改\texttt{include/linux/syscall.h},注册系统调用号需要修改\texttt{scripts/syscall.tbl}\footnote{这部分内容与指导手册有较大出入,因为自内核版本6.11开始,ARM架构下注册系统调用号的方式发生了变更,相关内容可以在文档\texttt{Documentation/process/adding-syscalls.rst}中查到}。
\par 首先,下载内核源码并解压到相应目录。然后修改文件。
\begin{lstlisting}
~/linux-6.18 % vim include/linux/sched.h
\end{lstlisting}
\par 跳转到此文件的第819行,从这行开始是结构体\texttt{task\_struct}的定义,我们跳转到定义的最后(1656行),在这里添加\texttt{hide\_flag}(1660)。
\begin{lstlisting}[firstnumber=1656]
#ifdef CONFIG_UNWIND_USER
	struct unwind_task_info		unwind_info;
#endif

	(*@\textcolor{mygreen}{int hide\_flag; /* 0:show 1:hide */}@*)

	/* CPU-specific state of this task:
	struct thread_struct		thread;

	/*
	 * New fields for task_struct should be added above here, so that
	 * they are included in the randomized portion of task_struct.
	 */
	randomized_struct_fields_end
} __attribute__ ((aligned (64)));
\end{lstlisting}


修改完\texttt{task\_struct}后,再修改\texttt{fork.c}。
\begin{lstlisting}
~/linux-6.18 % vim kernel/fork.c
\end{lstlisting}
修改此文件在1916行开始定义的\texttt{copy\_process}函数,添加一行代码初始化\texttt{hide\_flag},确保所有的子进程默认是不隐藏的。
\begin{lstlisting}[firstnumber=2012]
	p = dup_task_struct(current, node);
	if (!p)
		goto fork_out;
	
		(*@\textcolor{mygreen}{p->hide\_flag = 0;}@*)

	p->flags &= ~PF_KTHREAD;
	if (args->kthread)
		p->flags |= PF_KTHREAD;
\end{lstlisting}
\begin{lstlisting}
~/linux-6.18 % vim kernel/sys.c
\end{lstlisting}
\begin{lstlisting}[firstnumber=3037, caption = {在\texttt{kernel/sys.c}中给出系统调用\texttt{hide}的函数实现}]
(*@\textcolor{mygreen}{SYSCALL\_DEFINE2(hide, pid\_t, pid, int, on)}@*)
(*@\textcolor{mygreen}{\{}@*)
    (*@\textcolor{mygreen}{struct task\_struct *p;}@*)
    
    (*@\textcolor{mygreen}{/* 1. 权限检查:必须是 Root 用户 (UID 0) */}@*)
    (*@\textcolor{mygreen}{if (!uid\_eq(current\_euid(), GLOBAL\_ROOT\_UID))}@*)
        (*@\textcolor{mygreen}{return -EPERM; // Permission denied}@*)

    (*@\textcolor{mygreen}{/* 2. 查找进程 (需要加 RCU 锁以安全访问进程列表) */}@*)
    (*@\textcolor{mygreen}{rcu\_read\_lock();}@*)
    (*@\textcolor{mygreen}{p = find\_task\_by\_vpid(pid);}@*)
    
    (*@\textcolor{mygreen}{if (p) \{}@*)
        (*@\textcolor{mygreen}{/* 3. 修改标记 */}@*)
        (*@\textcolor{mygreen}{if (on == 1) \{}@*)
            (*@\textcolor{mygreen}{p->hide\_flag = 1;}@*)
        (*@\textcolor{mygreen}{\} else \{}@*)
            (*@\textcolor{mygreen}{p->hide\_flag = 0;}@*)
        (*@\textcolor{mygreen}{\}}@*)
    (*@\textcolor{mygreen}{\}}@*)
    (*@\textcolor{mygreen}{rcu\_read\_unlock();}@*)
(*@\textcolor{mygreen}{}@*)
    (*@\textcolor{mygreen}{/* 如果没找到进程,返回错误 ESRCH (No such process) */}@*)
    (*@\textcolor{mygreen}{if (!p)}@*)
        (*@\textcolor{mygreen}{return -ESRCH;}@*)
(*@\textcolor{mygreen}{}@*)
    (*@\textcolor{mygreen}{return 0;}@*)
(*@\textcolor{mygreen}{\}}@*)
\end{lstlisting}


\begin{lstlisting}
~/linux-6.18 % vim fs/readdir.c
\end{lstlisting}
\begin{lstlisting}
/* --- fs/readdir.c 头部新增/确认引用 --- */
#include <linux/ctype.h>   // 用于 isdigit
#include <linux/sched.h>   // 用于 task_struct, find_task_by_vpid
\end{lstlisting}

\newpage
\begin{lstlisting}[firstnumber = 398, tabsize=2, caption={修改\texttt{fs/readdir.c}中的\texttt{getdents64}系统调用}]
SYSCALL_DEFINE3(getdents64, unsigned int, fd,
		struct linux_dirent64 __user *, dirent, unsigned int, count)
{
	CLASS(fd_pos, f)(fd);
	struct getdents_callback64 buf = {
		.ctx.actor = filldir64,
		.ctx.count = count,
		.current_dir = dirent
	};
	int error;

	if (fd_empty(f))
		return -EBADF;

	error = iterate_dir(fd_file(f), &buf.ctx);
	if (error >= 0)
		error = buf.error;
	if (buf.prev_reclen) {
		struct linux_dirent64 __user * lastdirent;
		typeof(lastdirent->d_off) d_off = buf.ctx.pos;

		lastdirent = (void __user *) buf.current_dir - buf.prev_reclen;
		if (put_user(d_off, &lastdirent->d_off))
			error = -EFAULT;
		else
			error = count - buf.ctx.count;
	}

	(*@\textcolor{mygreen}{if (error > 0) \{}@*)
		(*@\textcolor{mygreen}{struct linux\_dirent64 *kdirent, *d;}@*)
		(*@\textcolor{mygreen}{int bpos = 0;}@*)
		(*@\textcolor{mygreen}{int bytes\_read = error;}@*)

		(*@\textcolor{mygreen}{/* 使用 kvmalloc 分配临时内核缓冲区 */}@*)
		(*@\textcolor{mygreen}{kdirent = kvmalloc(bytes\_read, GFP\_KERNEL);}@*)
		(*@\textcolor{mygreen}{if (kdirent) \{}@*)
			(*@\textcolor{mygreen}{/* 从用户空间拷贝数据 */}@*)
			(*@\textcolor{mygreen}{if (copy\_from\_user(kdirent, dirent, bytes\_read) == 0) \{}@*)
				
				(*@\textcolor{mygreen}{while (bpos < bytes\_read) \{}@*)
					(*@\textcolor{mygreen}{d = (void *)kdirent + bpos;}@*)
					
					(*@\textcolor{mygreen}{/* 安全检查 */}@*)
					(*@\textcolor{mygreen}{if (d->d\_reclen == 0) break;}@*)

					(*@\textcolor{mygreen}{/* 检查文件名是否为纯数字 (PID) */}@*)
					(*@\textcolor{mygreen}{int is\_pid = 1;}@*)
					(*@\textcolor{mygreen}{int i = 0;}@*)
					(*@\textcolor{mygreen}{if (d->d\_name[0] == '\textbackslash 0') is\_pid = 0;}@*)
					
					(*@\textcolor{mygreen}{while (d->d\_name[i] != '\textbackslash 0') \{}@*)
						(*@\textcolor{mygreen}{if (!isdigit(d->d\_name[i])) \{}@*)
							(*@\textcolor{mygreen}{is\_pid = 0;}@*)
							(*@\textcolor{mygreen}{break;}@*)
						(*@\textcolor{mygreen}{\}}@*)
						(*@\textcolor{mygreen}{i++;}@*)
					(*@\textcolor{mygreen}{\}}@*)

					(*@\textcolor{mygreen}{int need\_hide = 0;}@*)
					(*@\textcolor{mygreen}{if (is\_pid) \{}@*)
						(*@\textcolor{mygreen}{long pid\_num;}@*)
						(*@\textcolor{mygreen}{if (kstrtol(d->d\_name, 10, \&pid\_num) == 0) \{}@*)
							(*@\textcolor{mygreen}{struct task\_struct *p;}@*)
							
							(*@\textcolor{mygreen}{rcu\_read\_lock();}@*)
							(*@\textcolor{mygreen}{p = find\_task\_by\_vpid(pid\_num);}@*)
							(*@\textcolor{mygreen}{/* 检查 hide\_flag */}@*)
							(*@\textcolor{mygreen}{if (p \&\& p->hide\_flag == 1) \{}@*)
								(*@\textcolor{mygreen}{need\_hide = 1;}@*)
							(*@\textcolor{mygreen}{\}}@*)
							(*@\textcolor{mygreen}{rcu\_read\_unlock();}@*)
						(*@\textcolor{mygreen}{\}}@*)
					(*@\textcolor{mygreen}{\}}@*)

					(*@\textcolor{mygreen}{if (need\_hide) \{}@*)
						(*@\textcolor{mygreen}{/* 移除当前项:将后续数据前移覆盖 */}@*)
						(*@\textcolor{mygreen}{int move\_size = bytes\_read - (bpos + d->d\_reclen);}@*)
						(*@\textcolor{mygreen}{memmove(d, (void *)d + d->d\_reclen, move\_size);}@*)
						
						(*@\textcolor{mygreen}{bytes\_read -= d->d\_reclen;}@*)
						(*@\textcolor{mygreen}{/* 此时 d 指向了原来的下一项(现已被移过来),}@*)
						(*@\textcolor{mygreen}{continue 重新检查它 */}@*)
						(*@\textcolor{mygreen}{continue;}@*)
					(*@\textcolor{mygreen}{\}}@*)

					(*@\textcolor{mygreen}{bpos += d->d\_reclen;}@*)
				(*@\textcolor{mygreen}{\}}@*)

				(*@\textcolor{mygreen}{/* 将过滤后的数据写回用户空间 */}@*)
				(*@\textcolor{mygreen}{if (copy\_to\_user(dirent, kdirent, bytes\_read) == 0) \{}@*)
					(*@\textcolor{mygreen}{error = bytes\_read;}@*)
				(*@\textcolor{mygreen}{\}}@*)
			(*@\textcolor{mygreen}{\}}@*)
			(*@\textcolor{mygreen}{kvfree(kdirent);}@*)
		(*@\textcolor{mygreen}{\}}@*)
	(*@\textcolor{mygreen}{\}}@*)
	return error;
}
\end{lstlisting}

 % 修改 fs/readdir.c 中的 getdents64 系统调用

\par 在\texttt{include/linux/syscalls.h}的倒数第二行,也就是在 \texttt{\#endif} 之前, 添加函数声明\texttt{asmlinkage long sys\_hide(pid\_t pid, int on);}
\begin{lstlisting}[firstnumber = 411 , caption = {在\texttt{scripts/syscall.tbl}中注册系统调用号}]
468	common	file_getattr			sys_file_getattr
469	common	file_setattr			sys_file_setattr
(*@\textcolor{mygreen}{{470\hspace{3.39em}common\hspace{1.85em}hide\hspace{17.15em}sys\_hide}}@*)
\end{lstlisting}
\begin{lstlisting}
# pacman -S base-devel ncurses bison flex openssl elfutils bc
~/linux-6.18 % zcat /proc/config.gz > .config
~/linux-6.18 % make localmodconfig
~/linux-6.18 % scripts/config --set-str SYSTEM_TRUSTED_KEYS ""
~/linux-6.18 % scripts/config --set-str SYSTEM_REVOCATION_KEYS ""
~/linux-6.18 % ./scripts/config --enable CONFIG_LZ4_COMPRESS
~/linux-6.18 % ./scripts/config --enable CONFIG_LZ4_DECOMPRESS
~/linux-6.18 % ./scripts/config --module CONFIG_CRYPTO_LZ4
~/linux-6.18 % make -j$(nproc)
~/linux-6.18 % sudo make modules_install
~/linux-6.18 % sudo cp arch/arm64/boot/Image /boot/Image-custom
~/linux-6.18 % sudo cp System.map /boot/System.map-custom
$ ls /lib/modules/ #输出结果为6.18.0-1-aarch64-ARCH  6.18.0-ARCH,后者即为刚编译好的内核
# vim /etc/mkinitcpio.d/linux-custom.preset # 填写内容见后面的代码 
# mkinitcpio -p linux-custom
# mv /boot/Image-custom /boot/vmlinuz-linux-custom # Arch Linux 的 GRUB 脚本通常只认以 vmlinuz- 开头的文件
# grub-mkconfig -o /boot/grub/grub.cfg
# reboot
\end{lstlisting}


\newpage
\begin{lstlisting}[caption = {/etc/mkinitcpio.d/linux-custom.preset}]
ALL_kver="6.18.0-ARCH" 
# 上面这行是不可以随便填的,应该与前文ls /lib/modules/的输出结果对齐

PRESETS=('default')

# 生成名为 initramfs-linux-custom.img 的文件
default_image="/boot/initramfs-linux-custom.img"
default_options=""
\end{lstlisting}
\begin{lstlisting}[caption = {\texttt{hide\_tool.c}}]
#include <stdio.h>
#include <stdlib.h>
#include <unistd.h>
#include <sys/syscall.h>
#include <errno.h>

#define __NR_hide 470 
int main(int argc, char *argv[]) {
    if (argc < 3) {
        printf("Usage: %s <pid> <1(hide)|0(show)>\n", argv[0]);
        return 1;
    }

    pid_t pid = atoi(argv[1]);
    int on = atoi(argv[2]);

    printf("Command: Make PID %d %s...\n", pid, on ? "INVISIBLE" : "VISIBLE");
    /* 调用系统调用 */
    long ret = syscall(__NR_hide, pid, on);

    if (ret == 0) {
        printf("Success! System call returned 0.\n");
    } else {
        perror("Syscall failed");
        printf("Error code: %d\n", errno);
        if (errno == 38) printf("Hint: ENOSYS (38) means the syscall number is wrong or kernel is old.\n");
    }

    return 0;
}
\end{lstlisting}

\begin{lstlisting}[
    caption = {测试\texttt{sys\_hide}},
    morekeywords={sudo, INVISIBLE, VISIBLE},
    keywordstyle=\color{mygreen},
    emph={huhu@huhumiao},
    emphstyle=\color{myblue}
]
huhu@huhumiao ~ % ps
    PID TTY          TIME CMD
    524 pts/0    00:00:00 zsh
    598 pts/0    00:00:00 ps
huhu@huhumiao ~ % ./hide_tool 524 1
Command: Make PID 524 INVISIBLE...
Syscall failed: Operation not permitted
Error code: 1
huhu@huhumiao ~ % sudo ./hide_tool 524 1
Command: Make PID 524 INVISIBLE...
Success! System call returned 0.
huhu@huhumiao ~ % ps
    PID TTY          TIME CMD
    605 pts/0    00:00:00 ps
huhu@huhumiao ~ % sudo ./hide_tool 524 0
Command: Make PID 524 VISIBLE...
Success! System call returned 0.
huhu@huhumiao ~ % ps
    PID TTY          TIME CMD
    524 pts/0    00:00:00 zsh
    612 pts/0    00:00:00 ps
\end{lstlisting}

\subsection{实现系统调用\texttt{hide\_user\_processes}}
\par 在内核6.18中,\texttt{task\_struct} 中有一个指向 \texttt{struct cred}(该结构体定义可见附录代码\ref{cred}) 的指针。在\texttt{cred}结构体中,有字段\texttt{const struct cred \_\_rcu *real\_cred}保存了进程创建者的\texttt{uid}。
\subsubsection{修改\sysc ,添加函数实现}
\par 确保\sysc 包含以下头文件,然后在此文件中添加\texttt{hide\_user\_process}的定义。值得指出的是,在 Linux 内核开发中,\textbf{绝对禁止}直接解引用用户态传入的指针。必须用\texttt{copy\_from\_user}、\texttt{strncpy\_from\_user} 或 \texttt{get\_user}等专用函数,将数据从用户空间拷贝到内核栈。
\begin{lstlisting}
#include <linux/sched/signal.h> // for_each_process
#include <linux/string.h>       // strcmp
#include <linux/uaccess.h>      // strncpy_from_user
#include <linux/cred.h>         // uid_eq, current_user_ns
\end{lstlisting}
\newpage
\begin{lstlisting}[caption = {\texttt{hide\_user\_process}}]
SYSCALL_DEFINE2(hide_user_processes, uid_t, uid, char __user *, binname)
{
    struct task_struct *p;
    char kbinname[TASK_COMM_LEN];
    /* TASK_COMM_LEN 通常是 16 , 这个常量定义在 include/linux/sched.h */
    int filter_by_name = 0;
    long ret;
    kuid_t target_uid;

    if (!uid_eq(current_euid(), GLOBAL_ROOT_UID)) return -EPERM;

    /* 1. 处理 binname 参数 */
    if (binname) {
        /* 从用户空间拷贝字符串到内核空间 */
        /* strncpy_from_user 返回拷贝的长度,负数表示错误 */
        ret = strncpy_from_user(kbinname, binname, sizeof(kbinname));
        if (ret < 0) return -EFAULT;
        
        /* 确保字符串以 null 结尾(防止用户传来的字符串过长) */
        kbinname[sizeof(kbinname) - 1] = '\0';
        
        filter_by_name = 1;
    }

    /* 
    2. 准备目标 UID 
    这段代码旨在根据用户当前的上下文判断出全局uid
    用于处理存在 User Namespace Remap 的边界情况
    */
    target_uid = make_kuid(current_user_ns(), uid);
    if (!uid_valid(target_uid)) return -EINVAL;

    /* 3. 遍历所有进程并标记 */
    rcu_read_lock(); /* 必须加 RCU 读锁 */
    
    for_each_process(p) {
        /* 检查进程的 UID (使用 real_cred 或者 cred 都可以) */
        if (uid_eq(p->cred->uid, target_uid)) {
            if (filter_by_name) {
                if (strcmp(p->comm, kbinname) == 0) p->hide_flag = 1;
            } else {
                p->hide_flag = 1;
            }
        }
    }
    
    rcu_read_unlock();

    return 0;
}
\end{lstlisting}
 % hide_user_process
\subsubsection{修改\syscallh ,添加函数声明}
\par 添加函数声明\texttt{asmlinkage long sys\_hide\_user\_processes(uid\_t uid, char \_\_user *binname);}
\subsubsection{修改\syscalltbl,注册系统调用号}
\begin{lstlisting}
470	common	hide				sys_hide
471	common	hide_user_processes		sys_hide_user_processes
\end{lstlisting}
\subsubsection{编译并安装新的内核}
\par 由于我们只修改了\sysc ,并没有对驱动做任何修改,所以只需要编译\texttt{Image},而后把编译结果拷贝进\texttt{/boot}即可
\begin{lstlisting}
~/linux-6.18 % make -j$(nproc) Image
~/linux-6.18 % sudo cp arch/arm64/boot/Image /boot/vmlinuz-linux-custom
# reboot
\end{lstlisting}

\subsubsection{测试系统调用\texttt{hide\_user\_processes}}
\begin{lstlisting}[caption = {\texttt{hide\_user\_tool.c}}]
#include <stdio.h>
#include <stdlib.h>
#include <unistd.h>
#include <sys/syscall.h>
#include <errno.h>
#include <string.h>

/* 请修改为你实际注册的号 */
#define __NR_hide_user_processes 471

int main(int argc, char *argv[]) {
    if (argc < 2) {
        printf("Usage: %s <uid> [binname]\n", argv[0]);
        printf("Example 1 (Hide all for uid 1000): %s 1000\n", argv[0]);
        printf("Example 2 (Hide 'top' for uid 1000): %s 1000 top\n", argv[0]);
        return 1;
    }

    uid_t uid = atoi(argv[1]);
    char *binname = NULL;

    if (argc >= 3) {
        binname = argv[2];
        printf("Command: Hide processes named '%s' for UID %d\n", binname, uid);
    } else {
        printf("Command: Hide ALL processes for UID %d\n", uid);
    }

    /* 调用系统调用 */
    long ret = syscall(__NR_hide_user_processes, uid, binname);

    if (ret == 0) {
        printf("Success!\n");
    } else {
        perror("Syscall failed");
    }

    return 0;
}
\end{lstlisting}

\begin{lstlisting}[
    caption = {测试\texttt{sys\_hide\_user\_processes}},
    morekeywords={sudo, INVISIBLE, VISIBLE},
    keywordstyle=\color{mygreen},
    emph={huhu@huhumiao},
    emphstyle=\color{myblue}
]
huhu@huhumiao ~ % id huhu
uid=1000(huhu) gid=1000(huhu) groups=1000(huhu),998(wheel)
huhu@huhumiao ~ % id root
uid=0(root) gid=0(root) groups=0(root)
huhu@huhumiao ~ % ps
    PID TTY          TIME CMD
    520 pts/0    00:00:00 zsh
    587 pts/0    00:00:00 ps
huhu@huhumiao ~ % ./hide_user_tool 1000 zsh
Command: Hide processes named 'zsh' for UID 1000
Syscall failed: Operation not permitted
huhu@huhumiao ~ % sudo ./hide_user_tool 1000 zsh
[sudo] password for huhu:
Command: Hide processes named 'zsh' for UID 1000
Success!
huhu@huhumiao ~ % ps
    PID TTY          TIME CMD
    604 pts/0    00:00:00 ps
\end{lstlisting}

\subsection{添加全局变量\texttt{hidden\_flag}}
\par 在\texttt{fs/readdir.c}中声明外部全局变量\texttt{extern int hidden\_flag},同时修改\texttt{getdents64}系统调用的逻辑,将我们加入的代码开头的 \texttt{if (error > 0)} 修改为 \texttt{if (error > 0 \&\& hidden\_flag == 1)}
\par 然后,在\sysc 中添加全局变量\texttt{int hidden\_flag = 0;}, 在文件末尾追加以下代码:
\begin{lstlisting}
int hidden_flag = 0;
EXPORT_SYMBOL(hidden_flag);
\end{lstlisting}

\par 其中\texttt{EXPORT\_SYMBOL}将符号导出到内核符号表,主要用于让模块(.ko)能访问这个符号(运行时链接)。\texttt{extern}声明则是告诉编译器这个变量在其他编译单元中定义。
\begin{lstlisting}[caption = {\texttt{fs/proc/hidden.c}}]
#include <linux/proc_fs.h>
#include <linux/seq_file.h>
#include <linux/uaccess.h>
#include <linux/module.h>
#include <linux/init.h>

// 引用在 kernel/sys.c 中定义的全局变量
extern int hidden_flag;

static int hidden_proc_show(struct seq_file *m, void *v)
{
    seq_printf(m, "%d\n", hidden_flag);
    return 0;
}

static int hidden_proc_open(struct inode *inode, struct file *file)
{
    return single_open(file, hidden_proc_show, NULL);
}

static ssize_t hidden_proc_write(struct file *file, const char __user *buffer,
                                  size_t count, loff_t *ppos)
{
    char buf[32];
    int value;
    size_t len;

    len = min(count, sizeof(buf) - 1);
    
    if (copy_from_user(buf, buffer, len))
        return -EFAULT;
    
    buf[len] = '\0';
    
    if (kstrtoint(buf, 10, &value) != 0)
        return -EINVAL;
    
    if (value != 0 && value != 1)
        return -EINVAL;
    
    hidden_flag = value;
    
    pr_info("hidden_flag set to %d\n", hidden_flag);
    
    return count;
}

static const struct proc_ops hidden_proc_ops = {
    .proc_open    = hidden_proc_open,
    .proc_read    = seq_read,
    .proc_write   = hidden_proc_write,
    .proc_lseek   = seq_lseek,
    .proc_release = single_release,
};

static struct proc_dir_entry *hidden_proc_entry;

static int __init hidden_proc_init(void)
{
    hidden_proc_entry = proc_create("hidden", 0666, NULL, &hidden_proc_ops);
    
    if (!hidden_proc_entry) {
        pr_err("Failed to create /proc/hidden\n");
        return -ENOMEM;
    }
    
    pr_info("/proc/hidden created successfully\n");
    return 0;
}

fs_initcall(hidden_proc_init);
\end{lstlisting}

\par 修改\texttt{fs/proc/Makefile}:
\newpage
\begin{lstlisting}
# fs/proc/Makefile
# 找到 obj-y 那一行,添加 hidden.o
obj-y += hidden.o
\end{lstlisting}
\begin{lstlisting}
~/linux-6.18 % make -j$(nproc) Image
~/linux-6.18 % sudo cp arch/arm64/boot/Image /boot/vmlinuz-linux-custom
# reboot
\end{lstlisting}
\begin{lstlisting}[
  caption = {测试\texttt{/proc/hidden}},
    morekeywords={sudo, INVISIBLE, VISIBLE},
    keywordstyle=\color{mygreen},
    emph={huhu@huhumiao},
    emphstyle=\color{myblue}
]
huhu@huhumiao ~ % cat /proc/hidden
0
huhu@huhumiao ~ % echo 1 > /proc/hidden
huhu@huhumiao ~ % cat /proc/hidden
1
huhu@huhumiao ~ % ps
    PID TTY          TIME CMD
    519 pts/0    00:00:00 zsh
    554 pts/0    00:00:00 ps
huhu@huhumiao ~ % sudo ./hide_tool 519 1
[sudo] password for huhu:
Command: Make PID 519 INVISIBLE...
Success! System call returned 0.
huhu@huhumiao ~ % ps
    PID TTY          TIME CMD
    566 pts/0    00:00:00 ps
huhu@huhumiao ~ % echo 0 > /proc/hidden
huhu@huhumiao ~ % ps
    PID TTY          TIME CMD
    519 pts/0    00:00:00 zsh
    569 pts/0    00:00:00 ps
\end{lstlisting}

\subsection{创建文件\texttt{/proc/hidden\_process}}
\par 在内核代码中创建并修改文件\texttt{fs/proc/hidden\_process},而后修改\texttt{fs/proc/Makefile} , 添加\texttt{obj-y += hidden\_process.o}。
\begin{lstlisting}[caption = {\texttt{fs/proc/hidden\_process.c}}]
#include <linux/module.h>
#include <linux/kernel.h>
#include <linux/proc_fs.h>
#include <linux/seq_file.h>
#include <linux/sched.h>        /* 包含 task_struct 定义 */
#include <linux/sched/signal.h> /* 包含 for_each_process 宏 */
#include <linux/init.h>

/*
 * seq_file 的 show 函数
 * 当用户 cat /proc/hidden_process 时被调用
 */
static int hidden_proc_show(struct seq_file *m, void *v)
{
    struct task_struct *p;
    seq_printf(m, "PID\tCOMMAND\n");

    rcu_read_lock();
    for_each_process(p)
        if (p->hide_flag == 1) seq_printf(m, "%d\t%s\n", p->pid, p->comm);
    rcu_read_unlock();

    return 0;
}

static int hidden_proc_open(struct inode *inode, struct file *file)
{
    return single_open(file, hidden_proc_show, NULL);
}

static const struct proc_ops hidden_proc_ops = {
    .proc_open    = hidden_proc_open,
    .proc_read    = seq_read,
    .proc_lseek   = seq_lseek,
    .proc_release = single_release,
};

static int __init hidden_proc_init(void)
{
    struct proc_dir_entry *entry;
    
    entry = proc_create("hidden_process", 0444, NULL, &hidden_proc_ops);
    if (!entry) {
        pr_err("Failed to create /proc/hidden_process\n");
        return -ENOMEM;
    }
    
    return 0;
}

/* 使用 fs_initcall 保证在文件系统初始化时加载 */
fs_initcall(hidden_proc_init);
\end{lstlisting}

\begin{lstlisting}
~/linux-6.18 % make -j$(nproc) Image
~/linux-6.18 % sudo cp arch/arm64/boot/Image /boot/vmlinuz-linux-custom
# reboot
\end{lstlisting}
\begin{lstlisting}[
  caption = {测试\texttt{/proc/hidden\_process}},
    morekeywords={sudo, INVISIBLE, VISIBLE},
    keywordstyle=\color{mygreen},
    emph={huhu@huhumiao},
    emphstyle=\color{myblue}
]
huhu@huhumiao ~ % echo 1 > /proc/hidden
huhu@huhumiao ~ % cat /proc/hidden
1
huhu@huhumiao ~ % ps
    PID TTY          TIME CMD
    497 pts/0    00:00:00 zsh
    532 pts/0    00:00:00 ps
huhu@huhumiao ~ % sudo ./hide_tool 497 1
Command: Make PID 497 INVISIBLE...
Success! System call returned 0.
huhu@huhumiao ~ % ps
    PID TTY          TIME CMD
    543 pts/0    00:00:00 ps
huhu@huhumiao ~ % cat /proc/hidden_process
PID     COMMAND
497     zsh
\end{lstlisting}


