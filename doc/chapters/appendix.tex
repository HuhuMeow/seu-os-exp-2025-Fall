\newpage
\section{附录}
\subsection{实用工具}
\begin{lstlisting}
sudo pacman -S zoxide
# 若用 zsh , 在配置文件末尾添加
eval "$(zoxide init zsh)"
# 而后
source ~/.zshrc
\end{lstlisting}
\subsection{进程管理部分代码}
\begin{lstlisting}[caption = {\texttt{include/linux/cred.h}中\texttt{cred}的定义},label={cred}, firstnumber = 111, tabsize = 2]
struct cred {
	atomic_long_t	usage;
	kuid_t		uid;		/* real UID of the task */
	kgid_t		gid;		/* real GID of the task */
	kuid_t		suid;		/* saved UID of the task */
	kgid_t		sgid;		/* saved GID of the task */
	kuid_t		euid;		/* effective UID of the task */
	kgid_t		egid;		/* effective GID of the task */
	kuid_t		fsuid;		/* UID for VFS ops */
	kgid_t		fsgid;		/* GID for VFS ops */
	unsigned	securebits;	/* SUID-less security management */
	kernel_cap_t	cap_inheritable; /* caps our children can inherit */
	kernel_cap_t	cap_permitted;	/* caps we're permitted */
	kernel_cap_t	cap_effective;	/* caps we can actually use */
	kernel_cap_t	cap_bset;	/* capability bounding set */
	kernel_cap_t	cap_ambient;	/* Ambient capability set */
#ifdef CONFIG_KEYS
	unsigned char	jit_keyring;	/* default keyring to attach requested
					 * keys to */
	struct key	*session_keyring; /* keyring inherited over fork */
	struct key	*process_keyring; /* keyring private to this process */
	struct key	*thread_keyring; /* keyring private to this thread */
	struct key	*request_key_auth; /* assumed request_key authority */
#endif
#ifdef CONFIG_SECURITY
	void		*security;	/* LSM security */
#endif
	struct user_struct *user;	/* real user ID subscription */
	struct user_namespace *user_ns; /* user_ns the caps and keyrings are relative to. */
	struct ucounts *ucounts;
	struct group_info *group_info;	/* supplementary groups for euid/fsgid */
	/* RCU deletion */
	union {
		int non_rcu;			/* Can we skip RCU deletion? */
		struct rcu_head	rcu;		/* RCU deletion hook */
	};
} __randomize_layout;
\end{lstlisting}

\subsection{文件系统代码}
如果您需要使用这些代码的话,我并不建议您从此处复制代码,因为它有可能包含多余的空格和换行。 我建议前往此文档的GitHub仓库拉取代码。
文件系统的目录结构如下:
\dirtree{%
.1 fs.
.2 Makefile.
.2 image.
.2 mkfs.naive.
.2 mkfs.naive.c.
.2 mnt.
.2 naive\_fs.c.
.2 naive\_fs.h.
}
\par 其中,\texttt{image}是我们用\texttt{dd}程序生成的文件,\texttt{mnt}是用来挂载我们自己文件系统的目录。\texttt{mkfs.naive}是\texttt{mkfs.naive.c}编译生成的程序。

\begin{lstlisting}[caption = {\texttt{Makefile}}]
obj-m := naive_fs.o
KDIR := /lib/modules/$(shell uname -r)/build
PWD := $(shell pwd)

all:
	make -C $(KDIR) M=$(PWD) modules

clean:
	make -C $(KDIR) M=$(PWD) clean
\end{lstlisting}

\begin{lstlisting}[caption = {\texttt{mkfs.naive.c}}]
#include <stdio.h>
#include <stdlib.h>
#include <stdint.h>
#include <unistd.h>
#include <fcntl.h>
#include <string.h>
#include <time.h>
#include <sys/stat.h>

// 用户态兼容宏
#define __le32 uint32_t
#include "naive_fs.h"

int main(int argc, char *argv[]) {
    if (argc != 2) {
        printf("Usage: %s <device>\n", argv[0]);
        return 1;
    }

    int fd = open(argv[1], O_RDWR);
    if (fd < 0) {
        perror("Open failed");
        return 1;
    }

    // --- 步骤 1: 写入超级块 (Block 0) ---
    struct naive_super_block sb = {
        .magic = NAIVE_MAGIC,
        .block_count = 2000,
        .inode_count = 100,
        .root_inode = 1,
    };
    lseek(fd, 0, SEEK_SET);
    write(fd, &sb, sizeof(sb));

    // --- 步骤 2: 写入根目录 Inode (Block 1) ---
    // 我们的简化布局:
    // Block 0: Superblock
    // Block 1: Inode Table (包含 Root Inode)
    // Block 2: Root Directory Data
    
    struct naive_inode root_inode = {
        .mode = S_IFDIR | 0755,
        .size = sizeof(struct naive_dir_entry) * 2, // 初始只有 . 和 ..
        .blocks = 1,
        .data_block = { 2 } // 指向 Block 2
    };

    // 定位到 Block 1 的起始位置 + 1个inode偏移 (跳过inode 0)
    lseek(fd, NAIVE_BLOCK_SIZE + sizeof(struct naive_inode), SEEK_SET);
    write(fd, &root_inode, sizeof(root_inode));

    // --- 步骤 3: 写入根目录数据 (Block 2) ---
    struct naive_dir_entry entries[2];
    
    // 写入 "."
    memset(&entries[0], 0, sizeof(struct naive_dir_entry));
    strncpy(entries[0].name, ".", NAIVE_FILENAME_MAX);
    entries[0].inode_no = 1;

    // 写入 ".."
    memset(&entries[1], 0, sizeof(struct naive_dir_entry));
    strncpy(entries[1].name, "..", NAIVE_FILENAME_MAX);
    entries[1].inode_no = 1;

    lseek(fd, NAIVE_BLOCK_SIZE * 2, SEEK_SET);
    write(fd, entries, sizeof(entries));

    printf("NaiveFS formatted successfully on %s\n", argv[1]);
    printf("Magic Number: 0x%X\n", NAIVE_MAGIC);
    
    close(fd);
    return 0;
}
\end{lstlisting}

\input{chapters/lstlisting-code/naivefsc.tex}
\input{chapters/lstlisting-code/naivefsh.tex}
