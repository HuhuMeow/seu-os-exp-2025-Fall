\newpage
\section{shell}
\begin{quote}
The shell is an ordinary program that reads commands from the user and executes them. The fact that the shell is a user program, and not part of the kernel, illustrates the power of the system call interface: there is nothing special about the shell. It also means that the shell is easy to replace; as a result, modern Unix systems have a variety of shells to choose from, each with its own user interface and scripting features.\\
  \kaishu {shell 只是一个普通程序,负责读取并执行用户命令。正因为 shell 是用户程序而非内核的一部分,才凸显出系统调用接口的强大——shell 本身并无特殊之处。这也使得 shell 可以轻松替换;现代 Unix 系统正是如此,提供了多种 shell,各自拥有不同的用户界面和脚本功能。}
\par \hfill --xv6: a simple, Unix-like teaching operating system
\end{quote}
\begin{quote}
The shell is a command-line interpreter that prints a prompt, waits for you to type a command line, and then performs the command. If the first word of the command line does not correspond to a built-in shell command, then the shell assumes that it is the name of an executable file that it should load and run.\\
  \kaishu{shell 是一个命令行解释器,它会显示提示符,等待用户输入命令行,然后执行该命令。如果命令行的第一个单词不是内置的 shell 命令,shell 就会将其视为可执行文件的名称,并加载运行该文件。}
\par \hfill --Computer Systems: A Programmer's Perspective
\end{quote}
\subsection{实验内容}
实现具有\textbf{管道、重定向功能}的\texttt{shell},能够执行一些简单的基本命令,如进程执行、列目录等
\subsection{具体要求}
\begin{itemize}
  \item 设计一个\texttt{rust}语言程序\footnote{课程原本要求为C语言,但既然我们已经踏上了\texttt{Arch Linux}这条贼船,使用\texttt{rust}也是在情理之中的。},完成最基本的\texttt{shell}角色:给出命令行提示符、能够逐次接受命令。
对于命令分成三种
\begin{itemize}
  \item 内部命令(例如\texttt{help}命令、\texttt{exit}命令等)
  \item 外部命令(常见的\texttt{ls}、\texttt{cp}等,以及其他磁盘上的可执行程序\texttt{HelloWorld}等)
  \item 无效命令(不是上述二种命令)
\end{itemize}
\item 具有支持管道的功能,即在\texttt{shell}中输入诸如``\texttt{dir | more}''能够执行\texttt{dir}命令并将其输出通过管道将其输入传送给\texttt{more}。
\item 具有支持重定向的功能,即在\texttt{shell}中输入诸如``\texttt{dir > direct.txt}''能够执行\texttt{dir}命令并将结果输出到\texttt{direct.txt}
\item 将上述步骤直接合并完成
\end{itemize}
\subsection{环境配置}
在\texttt{Arch Linux}上配置\texttt{rust}开发环境只需两行
\begin{lstlisting}
sudo pacman -S rustup
rustup default stable
rustup component add rust-analyzer
\end{lstlisting}
可通过以下命令确认是否安装成功:
\begin{lstlisting}
rustc --version
cargo --version
\end{lstlisting}
用\texttt{cargo}创建项目:
\begin{lstlisting}
$ cargo new mysh
$ cd mysh
\end{lstlisting}
\subsection{实现基础功能}
在\texttt{mysh/src}目录下有一\texttt{main.rs}文件,对于我们的第一个版本的\texttt{shell},编辑这个文件就可以了。
\begin{lstlisting}[caption = {\texttt{main.rs}(ver1)}]
#[allow(unused_imports)]
use std::io::{self, Write};
use std::collections::HashSet;
use std::env;
use std::path::Path;
use std::fs;
use std::os::unix::fs::PermissionsExt;
use std::process::Command;

fn main() {
    let mut cmdline = String::new();
    let mut ok = true;
    let builtins: HashSet<&str> = HashSet::from(["exit","echo","type"]);
    let path = env::var("PATH").unwrap_or_default();
    let path_dirs: Vec<&str> = path.split(':').collect();
    let mut glob_exit_code;

    while ok{
        print!("$ ");
        io::stdout().flush().unwrap();
        io::stdin().read_line(&mut cmdline).unwrap();
        if cmdline.trim().is_empty() {
            cmdline.clear();
            continue;
        }
        let (cmd, argstr): (&str , &str) = 
        match cmdline.trim().split_once(' ') {
            Some((c, args)) => (c, args.trim()),
            None => (cmdline.trim(), ""),
        };
        match cmd{
            "exit" => ok = false,
            "echo" => println!("{}",argstr),
            "type" => {
                for arg in argstr.split_whitespace() {
                    if builtins.contains(arg) {
                        println!("{} is a shell builtin", arg);
                    }else if let Some(full_path) = 
                    find_executable(arg, &path_dirs){
                        println!("{} is {}", arg , full_path);
                    }else{
                        println!("{}: not found", arg); 
                    } 
                }
            }
            _ => {
                if let Some(full_path) = 
                find_executable(cmd, &path_dirs){
                    let status = 
                    Command::new(cmd)
                    .args(argstr.split_whitespace())
                    .status();
                    match status{
                        Ok(exit_code) => 
                        glob_exit_code = exit_code.code().unwrap_or(-1),
                        Err(e) => 
                        eprintln!("{}: execution error: {}", cmd, e),
                    }
                }else{
                    println!("{}: command not found", cmd)
                }
            }
        }
        cmdline.clear();
    }
}

fn find_executable(arg: &str , path_dirs : &[&str]) -> Option<String>{
    for dir in path_dirs {
        let full_path = format!("{}/{}", dir, arg);
        let path = Path::new(&full_path);
        
        if !path.exists() { continue; }
        
        if let Ok(metadata) = fs::metadata(path) {
            if metadata.permissions().mode() & 0o111 != 0 {
                return Some(full_path);
            }
        }
    }
    None
}
\end{lstlisting}

\par 这个版本的 \texttt{shell} 能够\textbf{处理内部命令和外部命令},但受限于简单的查找逻辑,暂时还无法执行磁盘上不在 \texttt{PATH} 路径内的程序(如显式指定相对路径 \texttt{./main})。 
\par \texttt{Rust} 的 \texttt{std::process::Command} 封装遮蔽了底层的复杂性:它会自动在 \texttt{PATH} 中搜索程序,并通过 \texttt{posix\_spawn} 或 \texttt{fork/exec} 组合来执行程序。在这个高层抽象下,开发者无需手动管理 \texttt{fork} 后的内存布局,也不必显式处理文件描述符(FD)的继承与关闭。而在 \texttt{C} 语言视角下,这个过程是完全手动的:
\par 当 \texttt{shell} 决定执行一个外部程序时,它首先调用 \texttt{fork}。此时,子进程完全\textbf{克隆}了父进程(Shell)的 \texttt{task\_struct},因此继承了完全一致的内存镜像(包含 \texttt{PATH} 变量)和文件描述符表(\texttt{files\_struct})。 \par 随后,子进程调用 \texttt{exec} 系统调用(如 \texttt{execve})。\texttt{exec} 会彻底销毁子进程原有的用户内存空间(栈、堆、代码段)。这意味着原有的 \texttt{PATH} 变量也会随之消失。因此,\texttt{shell} 必须在调用 \texttt{execve} 时,将 \texttt{PATH} 等环境变量作为参数(\texttt{envp})显式\textbf{传递}给内核。内核会将这些数据搬运到新程序的栈顶。 与内存不同,内核默认\textbf{保留}子进程的文件描述符表(除非设置了 \texttt{FD\_CLOEXEC})。这一特性至关重要:它允许 \texttt{shell} 在 \texttt{exec} 之前通过 \texttt{dup2} 调整标准输入输出指向(重定向),而新程序在完全不知情的情况下,自然继承这些指向文件或管道的文件描述符 ,从而实现 \texttt{I/O} 流的无缝衔接。
\subsection{重定向}
\par 在\texttt{Linux}中,每个进程都会预留3个默认的\texttt{fd}(file descriptor): \texttt{stdin}、\texttt{stdout}、\texttt{stderr};它们的值分别是0、1,2。在这一节,我们对代码进行了调整,仍然只有一个 \texttt{main.rs} 文件。\footnote{在下一节,我不得不重构本节的代码来实现下一节的需求。如果您在参考这份实验报告的话,推荐您先实现管道,再实现本节的重定向。若您期望直接参考成品,则第三节的代码为最终版本,不需要参考本节代码。}
\par 重定向所涉及到的关键字有六个,可分为四类。关键字分别为 \texttt{>} , \texttt{1>} , \texttt{>>} , \texttt{1>>} , \texttt{2>} , \texttt{2>>} . 其中,如果没有数字作为前缀,默认重定向\texttt{stdout}到文件,如果有的话,数字1代表重定向\texttt{stdout}, 数字2代表重定向\texttt{stderr}。符号 > 代表截断(Truncation)操作,即先清空文件,再写入;符号 \texttt{>>} 代表追加(Append)操作,在原文件的基础上继续写入。比如,\texttt{1>>} 代表重定向 \texttt{stdout} 到文件,将程序的输出\textbf{追加}到该文件后;\texttt{2>} 代表重定向 \texttt{stderr} 到文件,\textbf{清空}此文件(截断),然后再将输出写入此文件。
\par 在\texttt{rust}中,对程序进行重定向是简单的,不需要直接操纵文件描述符。在C语言中,则需要用\texttt{dup2}将文件描述符1或者2指向我们期望的文件。
\par 下面这个版本的\texttt{shell}在支持重定向的基础上,添加了对非\texttt{PATH}中程序的支持。可以通过绝对路径或者相对路径运行程序。
\begin{lstlisting}[caption = {\texttt{main.rs}(ver2)}]
#[allow(unused_imports)]
use std::io::{self, Write};
use std::collections::HashSet;
use std::env;
use std::path::Path;
use std::fs;
use std::fs::File;
use std::os::unix::fs::PermissionsExt;
use std::process::{Command,Stdio};

#[derive(PartialEq)]
enum RedirectKind {
    StdoutAppend,
    StderrAppend,
    StdoutTruncate,
    StderrTruncate,
}

fn main() {
    let mut cmdline = String::new();
    let mut ok = true;
    let builtins: HashSet<&str> = HashSet::from(["exit","echo","type"]);
    let path = env::var("PATH").unwrap_or_default();
    let path_dirs: Vec<&str> = path.split(':').collect();
    let mut glob_exit_code;

    while ok{
        print!("$ ");
        io::stdout().flush().unwrap();
        io::stdin().read_line(&mut cmdline).unwrap();
        if cmdline.trim().is_empty() {
            cmdline.clear();
            continue;
        }

        let (command_line, redirect_info) = 
        parse_redirection(cmdline.trim());

        let (cmd, argstr): (&str , &str) = 
        match command_line.trim().split_once(' ') {
            Some((c, args)) => (c, args.trim()),
            None => (command_line.trim(), ""),
        };

        match cmd{
            "exit" => ok = false,
            "echo" => {
                let argstr : String = 
                argstr.chars()
                .filter(|&c| c != '\"' && c != '\'')
                .collect();
                if let Some((ref file_path , ref kind)) = redirect_info{
                    let mut options = std::fs::OpenOptions::new();
                    options.write(true).create(true);
                    match *kind{
                        RedirectKind::StdoutTruncate | 
                        RedirectKind::StderrTruncate => 
                        {options.truncate(true);}
                        RedirectKind::StdoutAppend | 
                        RedirectKind::StderrAppend => 
                        {options.append(true);}
                    }
                    match options.open(file_path) {
                        Ok(mut file) => {
                            if *kind == RedirectKind::StderrAppend 
                            || *kind == RedirectKind::StderrTruncate {
                                println!("{}", argstr);
                            }else if let Err(e) 
                            = writeln!(file, "{}", argstr) {
                                eprintln!("Error writing to file: {}", e);
                            }
                        }
                        Err(e) => {
                            eprintln!("Error opening file {}: {}", file_path, e);
                            cmdline.clear();
                            continue;
                        }
                    }
                }else{
                    println!("{}", argstr);
                }
            }
            "type" => {
                for arg in argstr.split_whitespace() {
                    if builtins.contains(arg) {
                        println!("{} is a shell builtin", arg);
                    }else if let Some(full_path) 
                    = find_executable(arg, &path_dirs){
                        println!("{} is {}", arg , full_path);
                    }else{
                        println!("{}: not found", arg); 
                    } 
                }
            }
            _ => {
                let full_path 
                = if cmd.contains('/') || cmd.contains('\\') {
                    check_executable_path(cmd)
                } else {
                    find_executable(cmd, &path_dirs)
                };

                if let Some(full_path) = full_path{
                    let mut command = Command::new(full_path);
                    command.args(argstr.split_whitespace());

                    if let Some((ref file_path , ref kind)) 
                    = redirect_info{
                        let mut options = std::fs::OpenOptions::new();
                        options.write(true).create(true);
                        match *kind{
                            RedirectKind::StdoutTruncate 
                            | RedirectKind::StderrTruncate 
                            => {options.truncate(true);}
                            RedirectKind::StdoutAppend 
                            | RedirectKind::StderrAppend 
                            => {options.append(true);}
                        }

                        match options.open(file_path){
                            Ok(file) => { 
                                match *kind{
                                    RedirectKind::StderrTruncate 
                                    | RedirectKind::StderrAppend 
                                    => {command.stderr(Stdio::from(file));}
                                    RedirectKind::StdoutTruncate 
                                    | RedirectKind::StdoutAppend 
                                    => {command.stdout(Stdio::from(file));}
                                }
                            }
                            Err(e) => {
                                eprintln!(
                                "Error creating file {}: {}",
                                file_path,
                                e
                                );
                                cmdline.clear();
                                continue;
                            }
                        }
                    }
                    match command.status() {
                        Ok(exit_code) => 
                        glob_exit_code = exit_code.code().unwrap_or(-1),
                        Err(e) => 
                        eprintln!("{}: execution error: {}", cmd, e),
                    }
                } else {
                    let error_msg = format!("{}: command not found", cmd);
                    if let Some((ref file_path, 
                    RedirectKind::StderrTruncate)) = 
                    redirect_info {
                        write_to_file(file_path, &error_msg);
                    } else {
                        println!("{}", error_msg);
                    }
                }
            }
        }
        cmdline.clear();
    }
}

fn check_executable_path(path_str: &str) -> Option<String> {
    let path = Path::new(path_str);
    if !path.exists() { return None; }

    if let Ok(metadata) = fs::metadata(path) {
        if metadata.permissions().mode() & 0o111 != 0 {
            return Some(path_str.to_string());
        }
    }
    
    None
}

fn find_executable(arg: &str , path_dirs : &[&str]) -> Option<String>{
    for dir in path_dirs {
        let full_path = format!("{}/{}", dir, arg);
        let path = Path::new(&full_path);
        
        if !path.exists() { continue; }
        
        if let Ok(metadata) = fs::metadata(path) {
            if metadata.permissions().mode() & 0o111 != 0 {
                return Some(full_path);
            }
        }
    }
    None
}

fn parse_redirection(input: &str) 
-> (String, Option<(String, RedirectKind)>) {
    let redirect_patterns = [
        (" 2> ", RedirectKind::StderrTruncate),
        (" 1> ", RedirectKind::StdoutTruncate),
        (" > ", RedirectKind::StdoutTruncate),
        (" >> ", RedirectKind::StdoutAppend),
        (" 1>> ", RedirectKind::StdoutAppend),
        (" 2>> ", RedirectKind::StderrAppend),
    ];

    for (pattern, kind) in redirect_patterns {
        if let Some(pos) = input.find(pattern) {
            let cmd_part = input[..pos].trim().to_string();
            let file_part = 
            input[pos + pattern.len()..].trim().to_string();
            return (cmd_part, Some((file_part, kind)));
        }
    }

    (input.to_string(), None)
}

fn write_to_file(path: &str, content: &str) {
    match File::create(path) {
        Ok(mut file) => {
            if let Err(e) = writeln!(file, "{}", content) {
                eprintln!("Error writing to file: {}", e);
            }
        }
        Err(e) => eprintln!("Error creating file {}: {}", path, e),
    }
}

\end{lstlisting}

\subsection{管道}
\par 管道将串联多个命令的输入与输出,本质上与上一节的重定向并无区别。在解析命令时,我们的\texttt{shell}会先根据\texttt{|}符号将用户的输入分解为多段命令,对于第一个命令,我们重定向它的输出,但不会重定向它的输入\footnote{用户有把磁盘上文件输入进程序的需求,所以理论上应该支持重定向这个程序的输入部分,但为了保持我们的代码的简洁并照顾我的懒惰,我们的\texttt{shell}暂不支持这个功能。}; 对于最后一个命令,我们重定向它的输入,并检查是否有上一节提到的重定向操作符。
\par 前两个版本的\texttt{shell}代码不适合一次执行多条指令,因此对代码进行重构。
\begin{lstlisting}[caption = {\texttt{main.rs}(ver3)}]
#[allow(unused_imports)]
use std::io::{self, Write};
use std::collections::HashSet;
use std::env;
use std::path::Path;
use std::fs;
use std::fs::File;
use std::os::unix::fs::PermissionsExt;
use std::process::{Command, Stdio};

#[derive(PartialEq, Clone)]
enum RedirectKind {
    StdoutAppend,
    StderrAppend,
    StdoutTruncate,
    StderrTruncate,
}

struct ShellContext {
    builtins: HashSet<&'static str>,
    path_dirs: Vec<String>,
}

fn main() {
    let mut cmdline = String::new();
    let path = env::var("PATH").unwrap_or_default();
    let path_dirs: Vec<String> = 
    path.split(':').map(|s| s.to_string()).collect();
    
    let context = ShellContext {
        builtins: HashSet::from(["exit", "echo", "type"]),
        path_dirs,
    };

    loop {
        print!("$ ");
        io::stdout().flush().unwrap();
        
        cmdline.clear();
        io::stdin().read_line(&mut cmdline).unwrap();

        if cmdline.trim().is_empty() { continue; }
        
        if !execute_command_line(cmdline.trim(), &context) { 
            break; 
        }
    }
}

fn execute_command_line(cmdline: &str, context: &ShellContext) -> bool {
    if cmdline.contains('|') {
        let commands = parse_pipeline(cmdline);
        if let Err(e) = execute_pipeline(&commands, context) { 
            eprintln!("Pipeline error: {}", e); 
            }
        return true;
    }

    let (command_line, redirect_info) = parse_redirection(cmdline);
    
    let (cmd, argstr) = match command_line.trim().split_once(' ') {
        Some((c, args)) => (c, args.trim()),
        None => (command_line.trim(), ""),
    };

    if context.builtins.contains(cmd) { 
        return execute_builtin(cmd, 
        argstr, redirect_info.as_ref(), 
        context); 
    }
    execute_external_command(cmd, 
    argstr, redirect_info.as_ref(), 
    context);

    true
}


fn execute_builtin(
    cmd: &str,
    argstr: &str,
    redirect_info: Option<&(String, RedirectKind)>,
    context: &ShellContext,
) -> bool {
    match cmd {
        "exit" => false,
        "echo" => {
            builtin_echo(argstr, redirect_info);
            true
        }
        "type" => {
            builtin_type(argstr, context);
            true
        }
        _ => unreachable!("Unknown builtin command"),
    }
}

fn builtin_echo(
argstr: &str, 
redirect_info: Option<&(String, RedirectKind)>
) {
    let output: String = 
    argstr.chars()
    .filter(|&c| c != '\"' && c != '\'')
    .collect();
    
    if let Some((file_path, kind)) 
    = redirect_info {
        match open_redirect_file(file_path, kind) {
            Ok(mut file) => {
                if *kind == 
                RedirectKind::StderrAppend || 
                *kind == RedirectKind::StderrTruncate {
                    println!("{}", output);
                } else if let Err(e) = writeln!(file, "{}", output) {
                    eprintln!("Error writing to file: {}", e);
                }
            }
            Err(e) => { 
                eprintln!("Error opening file {}: {}", 
                file_path, e); 
            }
        }
    } else { println!("{}", output); }
}

fn builtin_type(argstr: &str, context: &ShellContext) {
    for arg in argstr.split_whitespace() {
        if context.builtins.contains(arg) {
            println!("{} is a shell builtin", arg);
        } else if let Some(full_path) 
        = find_executable(arg, &context.path_dirs) {
            println!("{} is {}", arg, full_path);
        } else {
            println!("{}: not found", arg);
        }
    }
}

fn execute_external_command(
    cmd: &str,
    argstr: &str,
    redirect_info: Option<&(String, RedirectKind)>,
    context: &ShellContext,
) {
    let full_path = 
    resolve_command_path(cmd, &context.path_dirs);

    match full_path {
        Some(path) => { 
            run_external_command(&path, 
            argstr, 
            redirect_info); 
        }
        None => { 
            let error_msg = format!("{}: command not found", cmd);
            println!("{}", error_msg);
        }
    }
}

fn run_external_command(
    full_path: &str,
    argstr: &str,
    redirect_info: Option<&(String, RedirectKind)>,
) {
    let mut command = Command::new(full_path);
    command.args(argstr.split_whitespace());

    if let Some((file_path, kind)) = redirect_info {
        match open_redirect_file(file_path, kind) {
            Ok(file) => {
                match kind {
                    RedirectKind::StderrTruncate | 
                    RedirectKind::StderrAppend => {
                        command.stderr(Stdio::from(file));
                    }
                    RedirectKind::StdoutTruncate | 
                    RedirectKind::StdoutAppend => {
                        command.stdout(Stdio::from(file));
                    }
                }
            }
            Err(e) => {
                eprintln!(
                    "Error opening file {}: {}", 
                    file_path, 
                    e);
                return;
            }
        }
    }

    match command.status() {
        Ok(_exit_status) => {}
        Err(e) => { eprintln!("Execution error: {}", e);}
    }
}

fn resolve_command_path(cmd: &str, path_dirs: &[String])
-> Option<String> {
    if cmd.contains('/') || cmd.contains('\\') {
        check_executable_path(cmd)
    } else {
        find_executable(cmd, path_dirs)
    }
}

fn check_executable_path(path_str: &str) 
-> Option<String> {
    let path = Path::new(path_str);
    if !path.exists() {
        return None;
    }

    if let Ok(metadata) = fs::metadata(path) {
        if metadata.permissions().mode() & 0o111 != 0 {
            return Some(path_str.to_string());
        }
    }

    None
}

fn find_executable(arg: &str, path_dirs: &[String]) 
-> Option<String> {
    for dir in path_dirs {
        let full_path = format!("{}/{}", dir, arg);
        let path = Path::new(&full_path);

        if !path.exists() {
            continue;
        }

        if let Ok(metadata) = fs::metadata(path) {
            if metadata.permissions().mode() & 0o111 != 0 {
                return Some(full_path);
            }
        }
    }
    None
}


fn parse_redirection(input: &str) 
-> (String, Option<(String, RedirectKind)>) {
    let redirect_patterns = [
        (" 2>> ", RedirectKind::StderrAppend),
        (" 1>> ", RedirectKind::StdoutAppend),
        (" >> ", RedirectKind::StdoutAppend),
        (" 2> ", RedirectKind::StderrTruncate),
        (" 1> ", RedirectKind::StdoutTruncate),
        (" > ", RedirectKind::StdoutTruncate),
    ];

    for (pattern, kind) in redirect_patterns {
        if let Some(pos) = input.find(pattern) {
            let cmd_part = input[..pos].trim().to_string();
            let file_part = 
            input[pos + pattern.len()..].trim().to_string();
            return (cmd_part, Some((file_part, kind)));
        }
    }

    (input.to_string(), None)
}

fn open_redirect_file(path: &str, kind: &RedirectKind) 
-> io::Result<File> {
    let mut options = fs::OpenOptions::new();
    options.write(true).create(true);

    match kind {
        RedirectKind::StdoutTruncate | RedirectKind::StderrTruncate => {
            options.truncate(true);
        }
        RedirectKind::StdoutAppend | RedirectKind::StderrAppend => {
            options.append(true);
        }
    }

    options.open(path)
}

fn parse_pipeline(input: &str) -> Vec<String> {
    input
        .split('|')
        .map(|s| s.trim().to_string())
        .filter(|s| !s.is_empty())
        .collect()
}

fn execute_pipeline(commands: &[String], context: &ShellContext) 
-> io::Result<()> {
    if commands.is_empty() { return Ok(()); }

    if commands.len() == 1 { 
        return execute_single_command(&commands[0], context);
    }

    let mut previous_stdout: Option<std::process::ChildStdout> = None;
    let mut children = Vec::new();

    for (i, cmd_str) in commands.iter().enumerate() {
        let (cmd_str, redirect_info) = if i == commands.len() - 1 {
            parse_redirection(cmd_str)
        } else {
            (cmd_str.to_string(), None)
        };

        let parts: Vec<&str> = cmd_str.split_whitespace().collect();
        if parts.is_empty() { continue; }

        let cmd = parts[0];
        let args = &parts[1..];

        let full_path = resolve_command_path(cmd, &context.path_dirs);

        let Some(full_path) = full_path else {
            eprintln!("{}: command not found", cmd);
            return Err(io::Error::new(
                io::ErrorKind::NotFound,
                "command not found",
            ));
        };

        let mut command = Command::new(full_path);
        command.args(args);

        if let Some(prev_stdout) 
        = previous_stdout.take() { command.stdin(prev_stdout);}

        if i == commands.len() - 1 {
            if let Some((file_path, kind)) = redirect_info {
                match open_redirect_file(&file_path, &kind) {
                    Ok(file) => match kind {
                        RedirectKind::StderrTruncate 
                        | RedirectKind::StderrAppend => {
                            command.stderr(Stdio::from(file));
                        }
                        RedirectKind::StdoutTruncate 
                        | RedirectKind::StdoutAppend => {
                            command.stdout(Stdio::from(file));
                        }
                    },
                    Err(e) => {
                        eprintln!(
                        "Error opening file {}: {}", file_path, e
                        );
                        return Err(e);
                    }
                }
            }
        } else {
            command.stdout(Stdio::piped());
        }

        let mut child = command.spawn()?;

        if i < commands.len() - 1 {
            previous_stdout = child.stdout.take();
        }

        children.push(child);
    }

    for mut child in children {
        child.wait()?;
    }

    Ok(())
}

fn execute_single_command(cmd_str: &str, context: &ShellContext) 
-> io::Result<()> {
    let parts: Vec<&str> = cmd_str.split_whitespace().collect();
    if parts.is_empty() {
        return Ok(());
    }

    let cmd = parts[0];
    let args = &parts[1..];

    let full_path =
    resolve_command_path(cmd, &context.path_dirs);

    let Some(full_path) = full_path else {
        eprintln!("{}: command not found", cmd);
        return Err(io::Error::new(
            io::ErrorKind::NotFound,
            "command not found",
        ));
    };

    Command::new(full_path).args(args).status()?;

    Ok(())
}

\end{lstlisting}

\subsection{局限}
\par 该\texttt{shell}仅仅是课程作业,一个像样的\texttt{shell}至少会支持自动补全、历史记录、切换目录等功能,更不必说\texttt{zsh}和\texttt{bash}这些庞然大物所支持的通配符、脚本语法、极强兼容性等外星科技。对于本\texttt{shell}已经实现的管道和重定向功能也是不完善的,如果您认真拆解了代码的话, 就会发现, 对于 \texttt{echo 1 | wc} 这种命令,我们的\texttt{shell}调用的并不是\texttt{builtin}命令,而是\texttt{PATH}中的后备隐藏能源。我们的\texttt{shell}也不支持重定向命令的输入。
\par 从代码实现顺序上来看,我们应该先实现管道,再实现重定向。这样是比较符合运算符优先级的,因为重定向一般出现在开头或者最后。
\par 尽管我们的\texttt{shell}有诸多的局限,但对于课程作业而言已经足够。我们有效掌握了利用\texttt{rust}解析路径,创建子进程,对进程重定向的技巧。
