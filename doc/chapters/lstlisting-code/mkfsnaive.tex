\begin{lstlisting}[caption = {\texttt{mkfs.naive.c}}]
#include <stdio.h>
#include <stdlib.h>
#include <stdint.h>
#include <unistd.h>
#include <fcntl.h>
#include <string.h>
#include <time.h>
#include <sys/stat.h>

// 用户态兼容宏
#define __le32 uint32_t
#include "naive_fs.h"

int main(int argc, char *argv[]) {
    if (argc != 2) {
        printf("Usage: %s <device>\n", argv[0]);
        return 1;
    }

    int fd = open(argv[1], O_RDWR);
    if (fd < 0) {
        perror("Open failed");
        return 1;
    }

    // --- 步骤 1: 写入超级块 (Block 0) ---
    struct naive_super_block sb = {
        .magic = NAIVE_MAGIC,
        .block_count = 2000,
        .inode_count = 100,
        .root_inode = 1,
    };
    lseek(fd, 0, SEEK_SET);
    write(fd, &sb, sizeof(sb));

    // --- 步骤 2: 写入根目录 Inode (Block 1) ---
    // 我们的简化布局:
    // Block 0: Superblock
    // Block 1: Inode Table (包含 Root Inode)
    // Block 2: Root Directory Data
    
    struct naive_inode root_inode = {
        .mode = S_IFDIR | 0755,
        .size = sizeof(struct naive_dir_entry) * 2, // 初始只有 . 和 ..
        .blocks = 1,
        .data_block = { 2 } // 指向 Block 2
    };

    // 定位到 Block 1 的起始位置 + 1个inode偏移 (跳过inode 0)
    lseek(fd, NAIVE_BLOCK_SIZE + sizeof(struct naive_inode), SEEK_SET);
    write(fd, &root_inode, sizeof(root_inode));

    // --- 步骤 3: 写入根目录数据 (Block 2) ---
    struct naive_dir_entry entries[2];
    
    // 写入 "."
    memset(&entries[0], 0, sizeof(struct naive_dir_entry));
    strncpy(entries[0].name, ".", NAIVE_FILENAME_MAX);
    entries[0].inode_no = 1;

    // 写入 ".."
    memset(&entries[1], 0, sizeof(struct naive_dir_entry));
    strncpy(entries[1].name, "..", NAIVE_FILENAME_MAX);
    entries[1].inode_no = 1;

    lseek(fd, NAIVE_BLOCK_SIZE * 2, SEEK_SET);
    write(fd, entries, sizeof(entries));

    printf("NaiveFS formatted successfully on %s\n", argv[1]);
    printf("Magic Number: 0x%X\n", NAIVE_MAGIC);
    
    close(fd);
    return 0;
}
\end{lstlisting}
