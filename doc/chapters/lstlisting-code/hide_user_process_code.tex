\begin{lstlisting}[caption = {\texttt{hide\_user\_process}}]
SYSCALL_DEFINE2(hide_user_processes, uid_t, uid, char __user *, binname)
{
    struct task_struct *p;
    char kbinname[TASK_COMM_LEN];
    /* TASK_COMM_LEN 通常是 16 , 这个常量定义在 include/linux/sched.h */
    int filter_by_name = 0;
    long ret;
    kuid_t target_uid;

    if (!uid_eq(current_euid(), GLOBAL_ROOT_UID)) return -EPERM;

    /* 1. 处理 binname 参数 */
    if (binname) {
        /* 从用户空间拷贝字符串到内核空间 */
        /* strncpy_from_user 返回拷贝的长度,负数表示错误 */
        ret = strncpy_from_user(kbinname, binname, sizeof(kbinname));
        if (ret < 0) return -EFAULT;
        
        /* 确保字符串以 null 结尾(防止用户传来的字符串过长) */
        kbinname[sizeof(kbinname) - 1] = '\0';
        
        filter_by_name = 1;
    }

    /* 
    2. 准备目标 UID 
    这段代码旨在根据用户当前的上下文判断出全局uid
    用于处理存在 User Namespace Remap 的边界情况
    */
    target_uid = make_kuid(current_user_ns(), uid);
    if (!uid_valid(target_uid)) return -EINVAL;

    /* 3. 遍历所有进程并标记 */
    rcu_read_lock(); /* 必须加 RCU 读锁 */
    
    for_each_process(p) {
        /* 检查进程的 UID (使用 real_cred 或者 cred 都可以) */
        if (uid_eq(p->cred->uid, target_uid)) {
            if (filter_by_name) {
                if (strcmp(p->comm, kbinname) == 0) p->hide_flag = 1;
            } else {
                p->hide_flag = 1;
            }
        }
    }
    
    rcu_read_unlock();

    return 0;
}
\end{lstlisting}
