%% introduction.tex 
%% ------------------
%% Author: HuhuMeow (Huhu_Miao)
%% License: MIT
\section{实验环境简述}
\par 本次实验采用\texttt{Arch Linux ARM aarch64},内核版本为{Linux 6.18.0-1-aarch64-ARCH}。虚拟机版本为VMware Fusion 13.6.4。
\section{Linux进程管理及其扩展}
\subsection{实验内容}
\par 实现一个系统调用\texttt{hide},使得可以根据指定的参数隐藏进程,使用户无法使用\texttt{ps}或\texttt{top}观察到进程状态。
\subsection{具体要求}
\begin{enumerate}
  \item 实现系统调用\texttt{int hide(pid\_t pid, int on)},在进程\texttt{pid}有效的前提下,如果\texttt{on}置\texttt{1},进程被隐藏,用户无法通过\texttt{ps}或\texttt{top}观察到进程状态;如果\texttt{on}置\texttt{0}且此前为隐藏状态,则恢复正常状态。
  \item 考虑权限问题,只有\texttt{root}用户才能隐藏进程。 
  \item 设计一个新的系统调用\texttt{int hide\_user\_processes(uid\_t uid, char *binname)},参数\texttt{uid}为用户ID号,当\texttt{binname}参数为\texttt{NULL}时,隐藏该用户的所有进程;否则,隐藏二进制映像名为\texttt{binname}的用户进程。该系统调用应与\texttt{hide}系统调用共存。
  \item 在\texttt{/proc}目录下创建一个文件\texttt{/proc/hidden},该文件可读可写,对应一个全局变量\texttt{hidden\_flag},当\texttt{hidden\_flag}为0时,所有进程都无法隐藏,即便此前进程被\texttt{hide}系统调用要求隐藏。只有当\texttt{hidden\_flag}为1时,此前通过\texttt{hide}调用要求被屏蔽的进程才隐藏起来
  \item 在\texttt{/proc}目录下创建一个文件\texttt{/proc/hidden\_process},该文件的内容包含所有被隐藏进程的\texttt{pid},各\texttt{pid}之间用空格分开
\end{enumerate}
\newpage
\subsection{实验过程}
\subsubsection{实现系统调用\texttt{hide}}
\par 首先,下载内核源码并解压到相应目录。然后修改文件。
\begin{lstlisting}
~/linux-6.18 % vim include/linux/sched.h
\end{lstlisting}
\par 跳转到此文件的第819行,从这行开始是结构体\texttt{task\_struct}的定义,我们跳转到定义的最后,也就是1656行,在这里添加\texttt{hide\_flag}(1660)。
\begin{lstlisting}[firstnumber=1656,escapeinside={(*@}{@*)}]
#ifdef CONFIG_UNWIND_USER
	struct unwind_task_info		unwind_info;
#endif

	int hide_flag; /* 0:show 1:hide */

	/* CPU-specific state of this task: */
	struct thread_struct		thread;

	/*
	 * New fields for task_struct should be added above here, so that
	 * they are included in the randomized portion of task_struct.
	 */
	randomized_struct_fields_end
} __attribute__ ((aligned (64)));
\end{lstlisting}
修改完\texttt{task\_struct}后,再修改\texttt{fork.c}。
\begin{lstlisting}
~/linux-6.18 % vim kernel/fork.c
\end{lstlisting}
修改此文件在1916行开始定义的\texttt{copy\_process}函数,添加一行代码初始化\texttt{hide\_flag},确保所有的子进程默认是不隐藏的。
\begin{lstlisting}[firstnumber=2012]
	p = dup_task_struct(current, node);
	if (!p)
		goto fork_out;
	
	p->hide_flag = 0;

	p->flags &= ~PF_KTHREAD;
	if (args->kthread)
		p->flags |= PF_KTHREAD;
\end{lstlisting}
\begin{lstlisting}
~/linux-6.18 % vim kernel/sys.c
\end{lstlisting}
\begin{lstlisting}[firstnumber=3037]
SYSCALL_DEFINE2(hide, pid_t, pid, int, on)
{
    struct task_struct *p;
    
    /* 1. 权限检查:必须是 Root 用户 (UID 0) */
    if (!uid_eq(current_euid(), GLOBAL_ROOT_UID))
        return -EPERM; // Permission denied

    /* 2. 查找进程 (需要加 RCU 锁以安全访问进程列表) */
    rcu_read_lock();
    p = find_task_by_vpid(pid);
    
    if (p) {
        /* 3. 修改标记 */
        if (on == 1) {
            p->hide_flag = 1;
        } else {
            p->hide_flag = 0;
        }
    }
    rcu_read_unlock();

    /* 如果没找到进程,返回错误 ESRCH (No such process) */
    if (!p)
        return -ESRCH;

    return 0;
}
#endif /* CONFIG_COMPAT */
\end{lstlisting}
\begin{lstlisting}
~/linux-6.18 % vim fs/readdir.c
\end{lstlisting}
\begin{lstlisting}[firstnumber=396]
SYSCALL_DEFINE3(getdents64, unsigned int, fd,
		struct linux_dirent64 __user *, dirent, unsigned int, count)
{
	CLASS(fd_pos, f)(fd);
	struct getdents_callback64 buf = {
		.ctx.actor = filldir64,
		.ctx.count = count,
		.current_dir = dirent
	};
	int error;

	if (fd_empty(f))
		return -EBADF;

	error = iterate_dir(fd_file(f), &buf.ctx);
	if (error >= 0)
		error = buf.error;
	if (buf.prev_reclen) {
		struct linux_dirent64 __user * lastdirent;
		typeof(lastdirent->d_off) d_off = buf.ctx.pos;

		lastdirent = (void __user *) buf.current_dir - buf.prev_reclen;
		if (put_user(d_off, &lastdirent->d_off))
			error = -EFAULT;
		else
			error = count - buf.ctx.count;
	}
	return error;
}
\end{lstlisting}
\begin{lstlisting}
/* --- fs/readdir.c 头部新增/确认引用 --- */
#include <linux/ctype.h>   // 用于 isdigit
#include <linux/sched.h>   // 用于 task_struct, find_task_by_vpid
\end{lstlisting}
\begin{lstlisting}[firstnumber = 398, tabsize=4, caption={修改\texttt{fs/readdir.c}中的\texttt{getdents64}系统调用}]
SYSCALL_DEFINE3(getdents64, unsigned int, fd,
		struct linux_dirent64 __user *, dirent, unsigned int, count)
{
	CLASS(fd_pos, f)(fd);
	struct getdents_callback64 buf = {
		.ctx.actor = filldir64,
		.ctx.count = count,
		.current_dir = dirent
	};
	int error;

	if (fd_empty(f))
		return -EBADF;

	error = iterate_dir(fd_file(f), &buf.ctx);
	if (error >= 0)
		error = buf.error;
	if (buf.prev_reclen) {
		struct linux_dirent64 __user * lastdirent;
		typeof(lastdirent->d_off) d_off = buf.ctx.pos;

		lastdirent = (void __user *) buf.current_dir - buf.prev_reclen;
		if (put_user(d_off, &lastdirent->d_off))
			error = -EFAULT;
		else
			error = count - buf.ctx.count;
	}

	if (error > 0) {
		struct linux_dirent64 *kdirent, *d;
		int bpos = 0;
		int bytes_read = error;

		/* 使用 kvmalloc 分配临时内核缓冲区 */
		kdirent = kvmalloc(bytes_read, GFP_KERNEL);
		if (kdirent) {
			/* 从用户空间拷贝数据 */
			if (copy_from_user(kdirent, dirent, bytes_read) == 0) {
				
				while (bpos < bytes_read) {
					d = (void *)kdirent + bpos;
					
					/* 安全检查 */
					if (d->d_reclen == 0) break;

					/* 检查文件名是否为纯数字 (PID) */
					int is_pid = 1;
					int i = 0;
					if (d->d_name[0] == '\0') is_pid = 0;
					
					while (d->d_name[i] != '\0') {
						if (!isdigit(d->d_name[i])) {
							is_pid = 0;
							break;
						}
						i++;
					}

					int need_hide = 0;
					if (is_pid) {
						long pid_num;
						if (kstrtol(d->d_name, 10, &pid_num) == 0) {
							struct task_struct *p;
							
							rcu_read_lock();
							p = find_task_by_vpid(pid_num);
							/* 检查 hide_flag */
							if (p && p->hide_flag == 1) {
								need_hide = 1;
							}
							rcu_read_unlock();
						}
					}

					if (need_hide) {
						/* 移除当前项:将后续数据前移覆盖 */
						int move_size = bytes_read - (bpos + d->d_reclen);
						memmove(d, (void *)d + d->d_reclen, move_size);
						
						bytes_read -= d->d_reclen;
						/* 此时 d 指向了原来的下一项(现已被移过来),continue 重新检查它 */
						continue;
					}

					bpos += d->d_reclen;
				}

				/* 将过滤后的数据写回用户空间 */
				if (copy_to_user(dirent, kdirent, bytes_read) == 0) {
					error = bytes_read;
				}
			}
			kvfree(kdirent);
		}
	}
	return error;
}
\end{lstlisting}
\par 在\texttt{include/linux/syscalls.h}的倒数第二行,也就是在 \texttt{\#endif} 之前, 添加声明\texttt{asmlinkage long sys\_hide(pid\_t pid, int on);}
\begin{lstlisting}[firstnumber = 411 , caption = {在\texttt{scripts/syscall.tbl}中注册系统调用号}]
468	common	file_getattr			sys_file_getattr
469	common	file_setattr			sys_file_setattr
470	common	hide				sys_hide
\end{lstlisting}
\begin{lstlisting}
# pacman -S base-devel ncurses bison flex openssl elfutils bc
~/linux-6.18 % zcat /proc/config.gz > .config
~/linux-6.18 % make localmodconfig
~/linux-6.18 % scripts/config --set-str SYSTEM_TRUSTED_KEYS ""
~/linux-6.18 % scripts/config --set-str SYSTEM_REVOCATION_KEYS ""
~/linux-6.18 % ./scripts/config --enable CONFIG_LZ4_COMPRESS
~/linux-6.18 % ./scripts/config --enable CONFIG_LZ4_DECOMPRESS
~/linux-6.18 % ./scripts/config --module CONFIG_CRYPTO_LZ4
~/linux-6.18 % make -j$(nproc)
~/linux-6.18 % sudo make modules_install
~/linux-6.18 % sudo cp arch/arm64/boot/Image /boot/Image-custom
~/linux-6.18 % sudo cp System.map /boot/System.map-custom
$ ls /lib/modules/ #输出结果为6.18.0-1-aarch64-ARCH  6.18.0-ARCH,后者即为刚编译好的内核
# vim /etc/mkinitcpio.d/linux-custom.preset # 填写内容见后面的代码 
# mkinitcpio -p linux-custom
# mv /boot/Image-custom /boot/vmlinuz-linux-custom # Arch Linux 的 GRUB 脚本通常只认以 vmlinuz- 开头的文件
# grub-mkconfig -o /boot/grub/grub.cfg
# reboot
\end{lstlisting}
\begin{lstlisting}[caption = {/etc/mkinitcpio.d/linux-custom.preset}]
ALL_kver="6.18.0-ARCH" # 这行是不可以随便填的,应该与前文ls /lib/modules/的输出结果对齐

PRESETS=('default')

# 生成名为 initramfs-linux-custom.img 的文件
default_image="/boot/initramfs-linux-custom.img"
default_options=""
\end{lstlisting}
\begin{lstlisting}[caption = {\texttt{hide\_tool.c}}]
#include <stdio.h>
#include <stdlib.h>
#include <unistd.h>
#include <sys/syscall.h>
#include <errno.h>

#define __NR_hide 470 

int main(int argc, char *argv[]) {
    if (argc < 3) {
        printf("Usage: %s <pid> <1(hide)|0(show)>\n", argv[0]);
        return 1;
    }

    pid_t pid = atoi(argv[1]);
    int on = atoi(argv[2]);

    printf("Command: Make PID %d %s...\n", pid, on ? "INVISIBLE" : "VISIBLE");

    /* 调用系统调用 */
    long ret = syscall(__NR_hide, pid, on);

    if (ret == 0) {
        printf("Success! System call returned 0.\n");
    } else {
        perror("Syscall failed");
        printf("Error code: %d\n", errno);
        if (errno == 38) printf("Hint: ENOSYS (38) means the syscall number is wrong or kernel is old.\n");
    }

    return 0;
}
\end{lstlisting}
\begin{lstlisting}[caption = {测试\texttt{sys\_hide}}]
huhu@huhumiao ~ % ps
    PID TTY          TIME CMD
    524 pts/0    00:00:00 zsh
    598 pts/0    00:00:00 ps
huhu@huhumiao ~ % sudo ./hide_tool 524 1
Command: Make PID 524 INVISIBLE...
Success! System call returned 0.
huhu@huhumiao ~ % ps
    PID TTY          TIME CMD
    605 pts/0    00:00:00 ps
huhu@huhumiao ~ % sudo ./hide_tool 524 0
Command: Make PID 524 VISIBLE...
Success! System call returned 0.
huhu@huhumiao ~ % ps
    PID TTY          TIME CMD
    524 pts/0    00:00:00 zsh
    612 pts/0    00:00:00 ps
\end{lstlisting}
