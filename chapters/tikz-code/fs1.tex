\begin{figure}[htbp]
\centering
\begin{tikzpicture}[
    block/.style={draw, minimum width=2.5cm, minimum height=1.2cm, align=center, font=\small},
    arrow/.style={->, >=stealth, thick}
]
    % --- 绘制各个块 (基于我们的 C 代码逻辑) ---
    
    % Block 0: 代码中直接 write(fd, &sb, ...) 到开头
    \node[block, fill=blue!20] (sb) at (0, 0) {超级块\\Block 0\\(Magic: 0x114514)};
    
    % Block 1: 代码中 lseek 到 512 写入 Inode
    \node[block, fill=orange!20] (itable) at (3, 0) {Inode 表\\Block 1\\(包含 Root Inode)};
    
    % Block 2: 代码中 lseek 到 1024 写入 "." 和 ".."
    \node[block, fill=yellow!20] (rootdata) at (6, 0) {根目录数据\\Block 2\\(存 . 和 ..)};
    
    % Block 3+: 未使用区域
    \node[block, fill=gray!10, dashed] (free) at (9, 0) {空闲数据区\\Block 3+};
    
    % --- 标注连接关系 ---
    % \draw[arrow] (sb.east) -- (itable.west);
    \draw[arrow] (itable.east) -- (rootdata.west);
    \draw[arrow] (rootdata.east) -- (free.west);

    \node[below=0.2cm, red] at (sb.south) {\texttt{Offset 0}};
    \node[below=0.2cm, red] at (itable.south) {\texttt{Offset 512}};
    \node[below=0.2cm, red] at (rootdata.south) {\texttt{Offset 1024}};
    
\end{tikzpicture}
\caption{简易文件系统磁盘布局}
\label{fig:real-fs-layout}
\end{figure}
