\begin{lstlisting}[firstnumber=3037, caption = {在\texttt{kernel/sys.c}中给出系统调用\texttt{hide}的函数实现}]
(*@\textcolor{mygreen}{SYSCALL\_DEFINE2(hide, pid\_t, pid, int, on)}@*)
(*@\textcolor{mygreen}{\{}@*)
    (*@\textcolor{mygreen}{struct task\_struct *p;}@*)
    
    (*@\textcolor{mygreen}{/* 1. 权限检查:必须是 Root 用户 (UID 0) */}@*)
    (*@\textcolor{mygreen}{if (!uid\_eq(current\_euid(), GLOBAL\_ROOT\_UID))}@*)
        (*@\textcolor{mygreen}{return -EPERM; // Permission denied}@*)

    (*@\textcolor{mygreen}{/* 2. 查找进程 (需要加 RCU 锁以安全访问进程列表) */}@*)
    (*@\textcolor{mygreen}{rcu\_read\_lock();}@*)
    (*@\textcolor{mygreen}{p = find\_task\_by\_vpid(pid);}@*)
    
    (*@\textcolor{mygreen}{if (p) \{}@*)
        (*@\textcolor{mygreen}{/* 3. 修改标记 */}@*)
        (*@\textcolor{mygreen}{if (on == 1) \{}@*)
            (*@\textcolor{mygreen}{p->hide\_flag = 1;}@*)
        (*@\textcolor{mygreen}{\} else \{}@*)
            (*@\textcolor{mygreen}{p->hide\_flag = 0;}@*)
        (*@\textcolor{mygreen}{\}}@*)
    (*@\textcolor{mygreen}{\}}@*)
    (*@\textcolor{mygreen}{rcu\_read\_unlock();}@*)
(*@\textcolor{mygreen}{}@*)
    (*@\textcolor{mygreen}{/* 如果没找到进程,返回错误 ESRCH (No such process) */}@*)
    (*@\textcolor{mygreen}{if (!p)}@*)
        (*@\textcolor{mygreen}{return -ESRCH;}@*)
(*@\textcolor{mygreen}{}@*)
    (*@\textcolor{mygreen}{return 0;}@*)
(*@\textcolor{mygreen}{\}}@*)
\end{lstlisting}

