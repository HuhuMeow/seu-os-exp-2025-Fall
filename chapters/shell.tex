\newpage
\section{shell}
\begin{quote}
The shell is an ordinary program that reads commands from the user and executes them. The fact that the shell is a user program, and not part of the kernel, illustrates the power of the system call interface: there is nothing special about the shell. It also means that the shell is easy to replace; as a result, modern Unix systems have a variety of shells to choose from, each with its own user interface and scripting features.\\
shell 只是一个普通程序,负责读取并执行用户命令。正因为 shell 是用户程序而非内核的一部分,才凸显出系统调用接口的强大——shell 本身并无特殊之处。这也使得 shell 可以轻松替换;现代 Unix 系统正是如此,提供了多种 shell,各自拥有不同的用户界面和脚本功能。
\par \hfill --xv6: a simple, Unix-like teaching operating system
\end{quote}
\begin{quote}
The shell is a command-line interpreter that prints a prompt, waits for you to type a command line, and then performs the command. If the first word of the command line does not correspond to a built-in shell command, then the shell assumes that it is the name of an executable file that it should load and run.\\
shell 是一个命令行解释器,它会显示提示符,等待用户输入命令行,然后执行该命令。如果命令行的第一个单词不是内置的 shell 命令,shell 就会将其视为可执行文件的名称,并加载运行该文件。
\par \hfill --Computer Systems: A Programmer's Perspective
\end{quote}
\subsection{实验内容}
实现具有\textbf{管道、重定向功能}的\texttt{shell},能够执行一些简单的基本命令,如进程执行、列目录等
\subsection{具体要求}
\begin{itemize}
  \item 设计一个\texttt{rust}语言程序\footnote{课程原本要求为C语言,但既然我们已经踏上了\texttt{Arch Linux}这条贼船,使用\texttt{rust}也是在情理之中的。},完成最基本的\texttt{shell}角色:给出命令行提示符、能够逐次接受命令。
对于命令分成三种
\begin{itemize}
  \item 内部命令(例如\texttt{help}命令、\texttt{exit}命令等)
  \item 外部命令(常见的\texttt{ls}、\texttt{cp}等,以及其他磁盘上的可执行程序\texttt{HelloWorld}等)
  \item 无效命令(不是上述二种命令)
\end{itemize}
\item 具有支持管道的功能,即在\texttt{shell}中输入诸如``\texttt{dir | more}''能够执行\texttt{dir}命令并将其输出通过管道将其输入传送给\texttt{more}。
\item 具有支持重定向的功能,即在\texttt{shell}中输入诸如``\texttt{dir > direct.txt}''能够执行\texttt{dir}命令并将结果输出到\texttt{direct.txt}
\item 将上述步骤直接合并完成
\end{itemize}
\subsection{环境配置}
在\texttt{Arch Linux}上配置\texttt{rust}开发环境只需两行
\begin{lstlisting}
sudo pacman -S rustup
rustup default stable
\end{lstlisting}
可通过以下命令确认是否安装成功:
\begin{lstlisting}
rustc --version
cargo --version
\end{lstlisting}
用\texttt{cargo}创建项目:
\begin{lstlisting}
$ cargo new mysh
$ cd mysh
\end{lstlisting}
在\texttt{mysh/src}目录下有一\texttt{main.rs}文件,对于我们的第一个版本的\texttt{shell},编辑这个文件就可以了。
\begin{lstlisting}[caption = {\texttt{main.rs}(ver1)}]
#[allow(unused_imports)]
use std::io::{self, Write};
use std::collections::HashSet;
use std::env;
use std::path::Path;
use std::fs;
use std::os::unix::fs::PermissionsExt;
use std::process::Command;

fn main() {
    let mut cmdline = String::new();
    let mut ok = true;
    let builtins: HashSet<&str> = HashSet::from(["exit","echo","type"]);
    let path = env::var("PATH").unwrap_or_default();
    let path_dirs: Vec<&str> = path.split(':').collect();
    let mut glob_exit_code;

    while ok{
        print!("$ ");
        io::stdout().flush().unwrap();
        io::stdin().read_line(&mut cmdline).unwrap();
        if cmdline.trim().is_empty() {
            cmdline.clear();
            continue;
        }
        let (cmd, argstr): (&str , &str) = 
        match cmdline.trim().split_once(' ') {
            Some((c, args)) => (c, args.trim()),
            None => (cmdline.trim(), ""),
        };
        match cmd{
            "exit" => ok = false,
            "echo" => println!("{}",argstr),
            "type" => {
                for arg in argstr.split_whitespace() {
                    if builtins.contains(arg) {
                        println!("{} is a shell builtin", arg);
                    }else if let Some(full_path) = 
                    find_executable(arg, &path_dirs){
                        println!("{} is {}", arg , full_path);
                    }else{
                        println!("{}: not found", arg); 
                    } 
                }
            }
            _ => {
                if let Some(full_path) = 
                find_executable(cmd, &path_dirs){
                    let status = 
                    Command::new(cmd)
                    .args(argstr.split_whitespace())
                    .status();
                    match status{
                        Ok(exit_code) => 
                        glob_exit_code = exit_code.code().unwrap_or(-1),
                        Err(e) => 
                        eprintln!("{}: execution error: {}", cmd, e),
                    }
                }else{
                    println!("{}: command not found", cmd)
                }
            }
        }
        cmdline.clear();
    }
}

fn find_executable(arg: &str , path_dirs : &[&str]) -> Option<String>{
    for dir in path_dirs {
        let full_path = format!("{}/{}", dir, arg);
        let path = Path::new(&full_path);
        
        if !path.exists() { continue; }
        
        if let Ok(metadata) = fs::metadata(path) {
            if metadata.permissions().mode() & 0o111 != 0 {
                return Some(full_path);
            }
        }
    }
    None
}
\end{lstlisting}

\par 这个版本的 \texttt{shell} 能够\textbf{处理内部命令和外部命令},但受限于简单的查找逻辑,暂时还无法执行磁盘上不在 \texttt{PATH} 路径内的程序(如显式指定相对路径 \texttt{./main})。 
\par \texttt{Rust} 的 \texttt{std::process::Command} 封装遮蔽了底层的复杂性:它会自动在 \texttt{PATH} 中搜索程序,并通过 \texttt{posix\_spawn} 或 \texttt{fork/exec} 组合来执行程序。在这个高层抽象下,开发者无需手动管理 \texttt{fork} 后的内存布局,也不必显式处理文件描述符(FD)的继承与关闭。而在 \texttt{C} 语言视角下,这个过程是完全手动的:
\par 当 \texttt{shell} 决定执行一个外部程序时,它首先调用 \texttt{fork}。此时,子进程完全\textbf{克隆}了父进程(Shell)的 \texttt{task\_struct},因此继承了完全一致的内存镜像(包含 \texttt{PATH} 变量)和文件描述符表(\texttt{files\_struct})。 \par 随后,子进程调用 \texttt{exec} 系统调用(如 \texttt{execve})。\texttt{exec} 会彻底销毁子进程原有的用户内存空间(栈、堆、代码段)。这意味着原有的 \texttt{PATH} 变量也会随之消失。因此,\texttt{shell} 必须在调用 \texttt{execve} 时,将 \texttt{PATH} 等环境变量作为参数(\texttt{envp})显式\textbf{传递}给内核。内核会将这些数据搬运到新程序的栈顶。 与内存不同,内核默认\textbf{保留}子进程的文件描述符表(除非设置了 \texttt{FD\_CLOEXEC})。这一特性至关重要:它允许 \texttt{shell} 在 \texttt{exec} 之前通过 \texttt{dup2} 调整标准输入输出指向(重定向),而新程序在完全不知情的情况下,自然继承这些指向文件或管道的文件描述符 ,从而实现 \texttt{I/O} 流的无缝衔接。
