\newpage
\section{shell}
\begin{quote}
The shell is an ordinary program that reads commands from the user and executes them. The fact that the shell is a user program, and not part of the kernel, illustrates the power of the system call interface: there is nothing special about the shell. It also means that the shell is easy to replace; as a result, modern Unix systems have a variety of shells to choose from, each with its own user interface and scripting features.\\
shell 只是一个普通程序,负责读取并执行用户命令。正因为 shell 是用户程序而非内核的一部分,才凸显出系统调用接口的强大——shell 本身并无特殊之处。这也使得 shell 可以轻松替换;现代 Unix 系统正是如此,提供了多种 shell,各自拥有不同的用户界面和脚本功能。
\par \hfill --xv6: a simple, Unix-like teaching operating system
\end{quote}
\begin{quote}
The shell is a command-line interpreter that prints a prompt, waits for you to type a command line, and then performs the command. If the first word of the command line does not correspond to a built-in shell command, then the shell assumes that it is the name of an executable file that it should load and run.\\
shell 是一个命令行解释器,它会显示提示符,等待用户输入命令行,然后执行该命令。如果命令行的第一个单词不是内置的 shell 命令,shell 就会将其视为可执行文件的名称,并加载运行该文件。
\par \hfill --Computer Systems: A Programmer's Perspective
\end{quote}
\subsection{实验内容}
实现具有\textbf{管道、重定向功能}的\texttt{shell},能够执行一些简单的基本命令,如进程执行、列目录等
\subsection{具体要求}
\begin{itemize}
  \item 设计一个\texttt{rust}语言程序\footnote{课程原本要求为C语言,但既然我们已经踏上了\texttt{Arch Linux}这条贼船,使用\texttt{rust}也是在情理之中的。},完成最基本的\texttt{shell}角色:给出命令行提示符、能够逐次接受命令。
对于命令分成三种
\begin{itemize}
  \item 内部命令(例如\texttt{help}命令、\texttt{exit}命令等)
  \item 外部命令(常见的\texttt{ls}、\texttt{cp}等,以及其他磁盘上的可执行程序\texttt{HelloWorld}等)
  \item 无效命令(不是上述二种命令)
\end{itemize}
\item 具有支持管道的功能,即在\texttt{shell}中输入诸如``\texttt{dir | more}''能够执行\texttt{dir}命令并将其输出通过管道将其输入传送给\texttt{more}。
\item 具有支持重定向的功能,即在\texttt{shell}中输入诸如``\texttt{dir > direct.txt}''能够执行\texttt{dir}命令并将结果输出到\texttt{direct.txt}
\item 将上述步骤直接合并完成
\end{itemize}
\subsection{环境配置}
在\texttt{Arch Linux}上配置\texttt{rust}开发环境只需两行
\begin{lstlisting}
sudo pacman -S rustup
rustup default stable
\end{lstlisting}
可通过以下命令确认是否安装成功:
\begin{lstlisting}
rustc --version
cargo --version
\end{lstlisting}
